\documentclass{article}
\usepackage[utf8]{inputenc}
\usepackage{enumitem}
\usepackage{amsmath}
\usepackage{amssymb}
\usepackage[letterpaper, margin=1in]{geometry}

\begin{document}

\begin{center}
    \Large \textbf{Guía N° 8: Derivadas} 
\end{center}

\vspace{1em}

\textbf{Estudiante:} Daniel Joaquín Orjuela Holguín

\textbf{Código:} 2243134

\textbf{Docente:} Iván Vega

\textbf{Asignatura:} Cálcuo 1

\textbf{Fecha:} 12 de Abril de 2025
\vspace{2em}
\begin{center}
    \Large \textbf{Reglas de derivada}
\end{center}  
    Las reglas de las derivadas son:
    \begin{itemize}
        \item $(fg)^{\prime}(x) = f^{\prime}(x)g(x) + f(x)g^{\prime}(x)$
        \item $(f/g)^{\prime}(x) = \dfrac{f^{\prime}(x)g(x) - f(x)g^{\prime}(x)}{(g(x))^2}$
    \end{itemize}
\vspace{0.5em}

\begin{enumerate}[start=5, label=\textbf{\arabic*.}]
    \item Suponga que $f(5) = 1, f^{\prime}(5) = 6, g(5) = -3, g^{\prime}(5) = 2$. Encuentre los valores siguientes:
        \begin{enumerate}[label=\alph*)] % Subítems con letras
            \item $(fg)^{\prime}(5)$
            \item $(f/g)^{\prime}(5)$
            \item $(g/f)^{\prime}(5)$
        \end{enumerate}
        \vspace{0.5em}
        
    \textbf{Solucion: }

    \begin{enumerate}[label=\alph*)] 
        \item $(fg)^{\prime}(5)= [6*(-3)] + (1*2) = -16$
        \item $(f/g)^{\prime}(5) = \dfrac {[6*(-3)] - (1*2)}{(-3)^2} = \dfrac{-20}{9}$
        \item $(g/f)^{\prime}(5) = \dfrac{(2*1) - [(-3)*6]}{1^2} = {20}$
        \vspace{0.5em}
    \end{enumerate}

    \item Si $f(x) = e^x g(x)$, donde $g(0) = 2, g^{\prime}(0) = 5$. Hallar $f^{\prime}(0)$.
    \vspace{0.5em}
    \vspace{0.5em}

    \textbf{Solucion: }

    $f^{\prime}(0) =  (e^0*2) + (e^0*5) = 7$
    \vspace{0.5em}
    \item Si $f$ es una funcíon derivable, encuentre una expresíon para la derivada de cada una de las siguientes:
    \begin{enumerate}[label=\alph*)]
        \item $y = x^2f(x)$
        \item $y = \dfrac{x^2}{f(x)}$
    \end{enumerate}

    \textbf{Solucion: }
    \begin{enumerate}[label=\alph*)]
        \item $y^{\prime} = (2x)f(x) + (x^2)f^{\prime}(x)$
        \item $y^{\prime} = \dfrac{(2x)f(x) - (x^2)f^{\prime}(x)}{(f(x))^2}$
    \end{enumerate}
    \vspace{0.5em}
    
    \item ¿Cúantas rectas tangentes a la curva $y = x/(x+1)$ pasan por el punto $(1,2)$?¿En qué puntos toca la curva a estas rectas tangentes?
    
    \vspace{0.5em}
    \textbf{Solucion: } 

    Suponiendo que la recta tangente toca la curva en $x=a$:
    \begin{itemize}
        \item Sabemos que la tangencia se presentan en $(a,f(a))$
        \item La pendiente en ese punto será $f^{\prime}(a)$
    \end{itemize}
    La ecuación punto pendiente de la recta tangente sería:
    \[
        y - f(a) = f^{\prime}(a)(x-a)
    \]
    Derivamos...
    \begin{itemize}    
        \item  $y = x/(x+1)$
        \item  $y^{\prime} = \dfrac{(1)(x+1) - (x)(1)}{(x+1)^2} = \dfrac{1}{(x+1)^2}$
    \end{itemize}
    Usando el punto dado inicialmente, tenemos que:
    \[
        2 - \dfrac{a}{a+1} = \dfrac{1}{(a+1)^2}(1-a)
    \]
    \[
        \dfrac{2a+2-a}{a+1} = \dfrac{1-a}{(a+1)^2}
    \]
    \[
        \dfrac{a+2}{a+1} = \dfrac{1-a}{(a+1)^2}
    \]
    \[
        (a+2)(a+1) = (1-a)
    \]
    \[
        a^2 + 3a + 2 = 1-a
    \]
    \[
        a^2 + 4a + 1 = 0
    \]
    Usando la formula general de las ecuaciones cuadráticas:
    \[
        a = \dfrac{-b \pm \sqrt{b^2 - 4ac}}{2a}
    \]
    Tenemos que:
    \[
        a = \dfrac{-4 \pm \sqrt{16 - 4}}{2} = \dfrac{-4 \pm \sqrt{12}}{2} = -2 \pm \sqrt{3}
    \]
    \[
        a_1 = -2 + \sqrt{3} \text{ y } a_2 = -2 - \sqrt{3}
    \]
    Entonces, las rectas tangentes pasan por los puntos:
    \[
        P_1 = (-2+\sqrt{3}, \dfrac{-2+\sqrt{3}}{-1+\sqrt{3}}) \text{ y }  P_2 = (-2-\sqrt{3}, \dfrac{-2-\sqrt{3}}{-1-\sqrt{3}})
    \]
    Y racionalizando:
    \[
        P_1 = (-2+\sqrt{3},  \dfrac{1-\sqrt{3}}{2}) \text{ y }  P_2 = (-2-\sqrt{3}, \dfrac{1+\sqrt{3}}{2})
    \]
    \vspace{1em}
    \item Si $h(2) = 4$ y $h^{\prime}  = -3,$ encuentre:
    \[
        \left.\dfrac{d}{dx}\left[\dfrac{h(x)}{x}\right]\right|_{x=2}
    \]
    \textbf{Solucion: }

    Usando la regla de la derivada del cociente:
    \[
        \left.\dfrac{d}{dx}\left[\dfrac{h(x)}{x}\right]\right|_{x=2} = \dfrac{h^{\prime}(2)x - h(2)}{x^2} = \dfrac{-3*2 - 4}{2^2} = \dfrac{-10}{4} = -\dfrac{5}{2}
    \]
    \par\vspace{3em}
    $\blacksquare$ Derive las siguientes funciónes:
    \vspace{1em}
    \item $g(x) = \sqrt{x}-2e^x$
    
    \textbf{Solucion: }
    \begin{itemize}
        \item $\sqrt{x} = x^{1/2}$
        \item $g^{\prime}(x) = \dfrac{1}{2}x^{-1/2} - 2e^x$
        \item $g^{\prime}(x) = \dfrac{1}{2\sqrt{x}} - 2e^x$
    \end{itemize}
    \vspace{0.5em}
    \item $y = 4\pi^2$
    
    \textbf{Solucion: }

    Es una constante, por lo tanto:
    \begin{itemize}
        \item $y^{\prime} = 0$
    \end{itemize}
    \vspace{0.5em}
    \item $y = \dfrac{x^2+4x+3}{\sqrt{x}}$
    
    \vspace{0.5em}
    \textbf{Solucion: }
    \begin{itemize}
        \item $y = (x^2+4x+3)(x^{-1/2})$
        \item $y^{\prime} = (2x+4)(x^{-1/2}) + (x^2+4x+3)(-\dfrac{1}{2}x^{-3/2})$
        \item $y^{\prime} = \dfrac{(2x+4)}{\sqrt{x}} - \dfrac{(x^2+4x+3)}{2x^{3/2}}$
        \item $y^{\prime} = \dfrac{(2x+4)}{\sqrt{x}} - \dfrac{(x^2+4x+3)}{2\sqrt{x^3}}$
    \end{itemize}
    \vspace{0.5em}

    \item $v = t^2-\dfrac{1}{\sqrt[4]{t^3}}$
    
    \textbf{Solucion: }
    \begin{itemize}
        \item $v = t^2 - t^{-3/4}$
        \item $v^{\prime} = 2t - (-\dfrac{3}{4})t^{-7/4}$
        \item $v^{\prime} = 2t + \dfrac{3}{4t^{7/4}}$
        \item $v^{\prime} = 2t + \dfrac{3}{4\sqrt[4]{t^7}}$
    \end{itemize}
    \vspace{0.5em}

    \item $z = \dfrac{A}{y^{10}}+Be^y$
    
    \textbf{Solucion: }
    \begin{itemize}
        \item $z = Ay^{-10} + Be^y$
        \item $z^{\prime} = -10Ay^{-11} + Be^y$
        \item $z^{\prime} = -\dfrac{10A}{y^{11}} + Be^y$
    \end{itemize}
    \vspace{0.5em}
    
    \item $g(x) = \dfrac{3x-1}{2x+1}$
    
    \textbf{Solucion: }
    \begin{itemize}
        \item $g^{\prime}(x) = \dfrac{(3)(2x+1) - (3x-1)(2)}{(2x+1)^2}$
        \item $g^{\prime}(x) = \dfrac{(6x+3) - (6x-2)}{(2x+1)^2}$
    \end{itemize}
    \vspace{0.5em}
        
    \item $F(y) = \left(\dfrac{1}{y^2}-\dfrac{3}{y^4}\right)(y+5y3)$
    
    \textbf{Solucion: }
    \begin{itemize}
        \item $F(y) = (y^{-2}-3y^{-4})(y+5y^3)$
        \item $F^{\prime}(y) = (-2y^{-3}+12y^{-5})(y+5y^3) + (y^{-2}-3y^{-4})(1+15y^2)$
        \item $F^{\prime}(y) = \left(-\dfrac{2}{y^3}+\dfrac{12}{y^5}\right)(y+5y^3)+\left(\dfrac{1}{y^2}-\dfrac{3}{y^4}\right)(1+15y^2)$
    \end{itemize}
    \vspace{0.5em}
        
    \item $y = \dfrac{t^2}{3t^2-2t+1}$
    
    \textbf{Solucion: }
    \begin{itemize}
        \item $y^{\prime} = \dfrac{(2t)(3t^2-2t+1) - (t^2)(6t-2)}{(3t^2-2t+1)^2}$
        \item $y^{\prime} = \dfrac{(6t^3-4t^2+2t) - (6t^3-2t^2)}{(3t^2-2t+1)^2}$
    \end{itemize}
    \vspace{0.5em}
        
    \item $f(x) = \dfrac{x}{x+\frac{c}{x}}$
    
    \textbf{Solucion: }
    \begin{itemize}
        \item $f(x) = \dfrac{x^2}{x^2+c}$
        \item $f^{\prime}(x) = \dfrac{(2x)(x^2+c) - (x^2)(2x)}{(x^2+c)^2}$
        \item $f^{\prime}(x) = \dfrac{(2x^3+2cx-2x^3)}{(x^2+c)^2}$
        \item $f^{\prime}(x) = \dfrac{2cx}{(x^2+c)^2}$
    \end{itemize}
    \vspace{0.5em}
        
    \item $g(t) = t^3cos\,t$
    
    \textbf{Solucion: }
    \begin{itemize}
        \item $g^{\prime}(t) = (3t^2)(cos\,t) + (t^3)(-sin\,t)$
        \item $g^{\prime}(t) = 3t^2\,cos\,t - t^3sin\,t$
    \end{itemize}
    \vspace{0.5em}
        
    \item $y = \dfrac{x}{cos\,x}$
    
    \textbf{Solucion: }
    \begin{itemize}
        \item $y^{\prime} = \dfrac{(1)(cos\,x) - (x)(-sin\,x)}{(cos\,x)^2}$
        \item $y^{\prime} = \dfrac{cos\,x + xsin\,x}{(cos\,x)^2}$
    \end{itemize}
    \vspace{0.5em}
        
    \item $y = \dfrac{sec\,\theta}{1+sec\,\theta}$
    
    \textbf{Solucion: }
    \begin{itemize}
        \item $y^{\prime} = \dfrac{[(sec\,\theta \cdot tan\,\theta)(1+sec\,\theta)] - [(sec\,\theta)(sec\,\theta \cdot tan\,\theta)]}{(1+sec\,\theta)^2}$
        \item $y^{\prime} = \dfrac{sec\,\theta \cdot tan\,\theta}{(1+sec\,\theta)^2}$
    \end{itemize}
    \vspace{0.5em}
    \item $f(x) = xe^x\,csc\,x$  
    
    \textbf{Solucion: }

    Agrupamos los términos $ xe^x$, por lo cual definimos que:
    \begin{itemize}
        \item g(x) = $xe^x$
        \item h(x) = $csc\,x$
    \end{itemize}
    Redefinimos la función como:
    \[
        f(x) = g(x)h(x)
    \]
    Y usamos la regla de la cadena y del producto para derivar:
    \[
        f^{\prime}(x) = g^{\prime}(x)h(x) + g(x)h^{\prime}(x)
    \]
    De este modo, tenemos que:
    \begin{itemize}
        \item $f^{\prime}(x) = [(e^x)(csc\,x)] + [(xe^x)(-csc\,x\cdot cot\,x)]$
        \item $f^{\prime}(x) = e^x\,csc\,x - xe^x\,csc\,x\,cot\,x$
        \item $f^{\prime}(x) = e^x\,csc\,x(1 - x\,cot\,x)$
    \end{itemize}
    \vspace{0.5em}

    \end{enumerate}
\end{document}
